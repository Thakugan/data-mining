\documentclass[12pt]{article}

\usepackage[utf8]{inputenc}   % Required for inputting international characters
\usepackage[T1]{fontenc}      % Output font encoding for international characters
\usepackage[sfdefault]{roboto}  % Set font to Roboto
\usepackage[margin=1in]{geometry}  % Set margins to 1 in all around
\usepackage{amsmath}          % for \hookrightarrow
\usepackage{color}            % for \textcolor
\usepackage{listings}         % Allows me to include R code

\definecolor{dkgreen}{rgb}{0,0.6,0}
\definecolor{gray}{rgb}{0.5,0.5,0.5}
\definecolor{mauve}{rgb}{0.58,0,0.82}

% Configuring code syntax
\lstset{
  language=R,
  columns=fullflexible,
  breaklines=true,
  postbreak=\mbox{\textcolor{red}{$\hookrightarrow$}\space},
  basicstyle=\footnotesize,       % the size of the fonts that are used for the code
  numbers=left,                   % where to put the line-numbers
  numberstyle=\tiny\color{gray},  % the style that is used for the line-numbers
  stepnumber=1,                   % the step between two line-numbers. If it's 1, each line will be numbered
  numbersep=8pt,                  % how far the line-numbers are from the code
  backgroundcolor=\color{white},  % choose the background color. You must add \usepackage{color}
  showspaces=false,               % show spaces adding particular underscores
  showstringspaces=false,         % underline spaces within strings
  showtabs=false,                 % show tabs within strings adding particular underscores
  rulecolor=\color{black},        % if not set, the frame-color may be changed on line-breaks within not-black text (e.g. commens (green here))
  tabsize=2,                      % sets default tabsize to 2 spaces
  captionpos=t,                   % sets the caption-position to bottom
  breaklines=true,                % sets automatic line breaking
  breakatwhitespace=false,        % sets if automatic breaks should only happen at whitespace
  title=\lstname,                 % show the filename of files included with \lstinputlisting;
                                  % also try caption instead of title
  keywordstyle=\color{blue},      % keyword style
  commentstyle=\color{dkgreen},   % comment style
  stringstyle=\color{mauve}      % string literal style
}

\title{Exploring Federal Employment Data Over Presidential Terms}
\author{Jenn Le}

\begin{document}

  \maketitle

  \thispagestyle{empty}

  \begin{abstract}

    Add an abstract/executive summary (¼ page) that introduces the problem and
    highlights the major results. Be concrete with your major results (e.g., "This
    report shows that...").

  \end{abstract}

  \clearpage
  \pagenumbering{arabic}

  \tableofcontents
  \pagebreak

  \section{Business Understanding}
    Data Source: https://archive.org/details/opm-federal-employment-data
    \\
    \\
    This data contains federal employment information over the course of President
    George W. Bush's and President Barack Obama's terms in office.

  \section{Data Understanding}

    I examined the status data of government employees, excluding the Department
    of Defense, for the years 2001 - 2014. This includes Bush's entire presidency
    and all but the last two years of Obama's.

    The main code file I used is shown below and outlines the steps I took in my
    investigation of the data.
    \\
    \lstinputlisting{../code/project-code.R}

    \subsection{Collecting Data}
      Below is the function that I used to collect the text data for a specific year,
      concatenate the files for each quarter, and convert it into a dataframe. I
      used the headers files that was provided to label the columns in the dataframe.
      \\
      \lstinputlisting{../code/collect-data.R}

  \section{Data Preparation}

  \section{Modeling}

  \section{Evaluation}

\end{document}
